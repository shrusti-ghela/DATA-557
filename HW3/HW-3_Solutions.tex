% Options for packages loaded elsewhere
\PassOptionsToPackage{unicode}{hyperref}
\PassOptionsToPackage{hyphens}{url}
%
\documentclass[
]{article}
\title{Homework Assignment 2}
\author{Shrusti Ghela}
\date{February 09, 2022}

\usepackage{amsmath,amssymb}
\usepackage{lmodern}
\usepackage{iftex}
\ifPDFTeX
  \usepackage[T1]{fontenc}
  \usepackage[utf8]{inputenc}
  \usepackage{textcomp} % provide euro and other symbols
\else % if luatex or xetex
  \usepackage{unicode-math}
  \defaultfontfeatures{Scale=MatchLowercase}
  \defaultfontfeatures[\rmfamily]{Ligatures=TeX,Scale=1}
\fi
% Use upquote if available, for straight quotes in verbatim environments
\IfFileExists{upquote.sty}{\usepackage{upquote}}{}
\IfFileExists{microtype.sty}{% use microtype if available
  \usepackage[]{microtype}
  \UseMicrotypeSet[protrusion]{basicmath} % disable protrusion for tt fonts
}{}
\makeatletter
\@ifundefined{KOMAClassName}{% if non-KOMA class
  \IfFileExists{parskip.sty}{%
    \usepackage{parskip}
  }{% else
    \setlength{\parindent}{0pt}
    \setlength{\parskip}{6pt plus 2pt minus 1pt}}
}{% if KOMA class
  \KOMAoptions{parskip=half}}
\makeatother
\usepackage{xcolor}
\IfFileExists{xurl.sty}{\usepackage{xurl}}{} % add URL line breaks if available
\IfFileExists{bookmark.sty}{\usepackage{bookmark}}{\usepackage{hyperref}}
\hypersetup{
  pdftitle={Homework Assignment 2},
  pdfauthor={Shrusti Ghela},
  hidelinks,
  pdfcreator={LaTeX via pandoc}}
\urlstyle{same} % disable monospaced font for URLs
\usepackage[margin=1in]{geometry}
\usepackage{color}
\usepackage{fancyvrb}
\newcommand{\VerbBar}{|}
\newcommand{\VERB}{\Verb[commandchars=\\\{\}]}
\DefineVerbatimEnvironment{Highlighting}{Verbatim}{commandchars=\\\{\}}
% Add ',fontsize=\small' for more characters per line
\usepackage{framed}
\definecolor{shadecolor}{RGB}{248,248,248}
\newenvironment{Shaded}{\begin{snugshade}}{\end{snugshade}}
\newcommand{\AlertTok}[1]{\textcolor[rgb]{0.94,0.16,0.16}{#1}}
\newcommand{\AnnotationTok}[1]{\textcolor[rgb]{0.56,0.35,0.01}{\textbf{\textit{#1}}}}
\newcommand{\AttributeTok}[1]{\textcolor[rgb]{0.77,0.63,0.00}{#1}}
\newcommand{\BaseNTok}[1]{\textcolor[rgb]{0.00,0.00,0.81}{#1}}
\newcommand{\BuiltInTok}[1]{#1}
\newcommand{\CharTok}[1]{\textcolor[rgb]{0.31,0.60,0.02}{#1}}
\newcommand{\CommentTok}[1]{\textcolor[rgb]{0.56,0.35,0.01}{\textit{#1}}}
\newcommand{\CommentVarTok}[1]{\textcolor[rgb]{0.56,0.35,0.01}{\textbf{\textit{#1}}}}
\newcommand{\ConstantTok}[1]{\textcolor[rgb]{0.00,0.00,0.00}{#1}}
\newcommand{\ControlFlowTok}[1]{\textcolor[rgb]{0.13,0.29,0.53}{\textbf{#1}}}
\newcommand{\DataTypeTok}[1]{\textcolor[rgb]{0.13,0.29,0.53}{#1}}
\newcommand{\DecValTok}[1]{\textcolor[rgb]{0.00,0.00,0.81}{#1}}
\newcommand{\DocumentationTok}[1]{\textcolor[rgb]{0.56,0.35,0.01}{\textbf{\textit{#1}}}}
\newcommand{\ErrorTok}[1]{\textcolor[rgb]{0.64,0.00,0.00}{\textbf{#1}}}
\newcommand{\ExtensionTok}[1]{#1}
\newcommand{\FloatTok}[1]{\textcolor[rgb]{0.00,0.00,0.81}{#1}}
\newcommand{\FunctionTok}[1]{\textcolor[rgb]{0.00,0.00,0.00}{#1}}
\newcommand{\ImportTok}[1]{#1}
\newcommand{\InformationTok}[1]{\textcolor[rgb]{0.56,0.35,0.01}{\textbf{\textit{#1}}}}
\newcommand{\KeywordTok}[1]{\textcolor[rgb]{0.13,0.29,0.53}{\textbf{#1}}}
\newcommand{\NormalTok}[1]{#1}
\newcommand{\OperatorTok}[1]{\textcolor[rgb]{0.81,0.36,0.00}{\textbf{#1}}}
\newcommand{\OtherTok}[1]{\textcolor[rgb]{0.56,0.35,0.01}{#1}}
\newcommand{\PreprocessorTok}[1]{\textcolor[rgb]{0.56,0.35,0.01}{\textit{#1}}}
\newcommand{\RegionMarkerTok}[1]{#1}
\newcommand{\SpecialCharTok}[1]{\textcolor[rgb]{0.00,0.00,0.00}{#1}}
\newcommand{\SpecialStringTok}[1]{\textcolor[rgb]{0.31,0.60,0.02}{#1}}
\newcommand{\StringTok}[1]{\textcolor[rgb]{0.31,0.60,0.02}{#1}}
\newcommand{\VariableTok}[1]{\textcolor[rgb]{0.00,0.00,0.00}{#1}}
\newcommand{\VerbatimStringTok}[1]{\textcolor[rgb]{0.31,0.60,0.02}{#1}}
\newcommand{\WarningTok}[1]{\textcolor[rgb]{0.56,0.35,0.01}{\textbf{\textit{#1}}}}
\usepackage{graphicx}
\makeatletter
\def\maxwidth{\ifdim\Gin@nat@width>\linewidth\linewidth\else\Gin@nat@width\fi}
\def\maxheight{\ifdim\Gin@nat@height>\textheight\textheight\else\Gin@nat@height\fi}
\makeatother
% Scale images if necessary, so that they will not overflow the page
% margins by default, and it is still possible to overwrite the defaults
% using explicit options in \includegraphics[width, height, ...]{}
\setkeys{Gin}{width=\maxwidth,height=\maxheight,keepaspectratio}
% Set default figure placement to htbp
\makeatletter
\def\fps@figure{htbp}
\makeatother
\setlength{\emergencystretch}{3em} % prevent overfull lines
\providecommand{\tightlist}{%
  \setlength{\itemsep}{0pt}\setlength{\parskip}{0pt}}
\setcounter{secnumdepth}{-\maxdimen} % remove section numbering
\ifLuaTeX
  \usepackage{selnolig}  % disable illegal ligatures
\fi

\begin{document}
\maketitle

Data: `lead.csv'

The data are from a study of the association between exposure to lead
and IQ. The study was conducted in an urban area around a lead smelter.
A random sample of 124 children who lived in the area was selected. Each
study participant had a blood sample drawn in both 1972 and 1973 to
assess blood concentrations of lead. The children were grouped based on
their blood concentrations as follows:

Group 1: concentration \textless{} 40 mg/L in both 1972 and 1973 Group
2: concentration \textgreater{} 40 mg/L in both 1972 and 1973 or
\textgreater{} 40 mg/L in 1973 alone (3 participants) Group 3:
concentration \textgreater{} 40 mg/L in 1972 but \textless{} 40 mg/L in
1973

Each participant completed an IQ test in 1973. (A subset of the IQ
scores from this study were used in HW 1, Question 3.) The variables in
the data set are listed below.

ID: Participant identification number SEX: Participant sex (1=M or 2=F)
GROUP: As described above (1, 2, or 3) IQ: IQ score

\begin{Shaded}
\begin{Highlighting}[]
\NormalTok{lead }\OtherTok{\textless{}{-}} \FunctionTok{read.csv}\NormalTok{(}\StringTok{"lead\_study.csv"}\NormalTok{)}
\end{Highlighting}
\end{Shaded}

\textbf{1. The first goal is to compare the mean IQ scores for males and
females. Use a 2-sample t-test for this comparison. What is the
p-value?}

\begin{Shaded}
\begin{Highlighting}[]
\NormalTok{m }\OtherTok{=} \FunctionTok{with}\NormalTok{(lead, }\FunctionTok{tapply}\NormalTok{(IQ, SEX,mean))}
\NormalTok{s }\OtherTok{=} \FunctionTok{with}\NormalTok{(lead, }\FunctionTok{tapply}\NormalTok{(IQ, SEX,sd))}
\NormalTok{n }\OtherTok{=} \FunctionTok{with}\NormalTok{(lead, }\FunctionTok{tapply}\NormalTok{(IQ, SEX,length))}
\FunctionTok{data.frame}\NormalTok{(m,s,n)}
\end{Highlighting}
\end{Shaded}

\begin{verbatim}
##          m        s  n
## 1 91.23684 14.93083 76
## 2 90.87671 13.58507 73
\end{verbatim}

Now to compare the mean IQ squares for 1 and 2.

\begin{Shaded}
\begin{Highlighting}[]
\FunctionTok{with}\NormalTok{(lead, }\FunctionTok{t.test}\NormalTok{(IQ[SEX}\SpecialCharTok{==}\DecValTok{1}\NormalTok{], IQ[SEX}\SpecialCharTok{==}\DecValTok{2}\NormalTok{], }\AttributeTok{var.equal=}\NormalTok{T))}
\end{Highlighting}
\end{Shaded}

\begin{verbatim}
## 
##  Two Sample t-test
## 
## data:  IQ[SEX == 1] and IQ[SEX == 2]
## t = 0.15381, df = 147, p-value = 0.878
## alternative hypothesis: true difference in means is not equal to 0
## 95 percent confidence interval:
##  -4.267092  4.987351
## sample estimates:
## mean of x mean of y 
##  91.23684  90.87671
\end{verbatim}

\textbf{2. State the conclusion from your test.}

The p-value is greater than 0.05, we would not reject the null
hypothesis for equal means for males and females.

\textbf{3. Are the independence assumptions valid for the t-test in this
situation? Give a brief explanation.} Yes, the independence assumptions
would be valid in this situation as the data for males and females are
independent of each other, where data for one group would not affect the
other group.

\textbf{4. The second goal is to compare the mean IQ scores in the 3
groups. State in words the null hypothesis for this test.}

The null hypothesis for this case could be defined as the mean IQ scoreS
for the 3 groups are equal.

\textbf{5. State in words the alternative hypothesis for this test.}

We define the alternative hypothesis as: Not all mean IQ scores for 3
groups are equal.

\textbf{6. What method should be used to perform the test?}

ANOVA ANOVA is simply an extension of the 2-sample equal-variance t-test
to the comparison of 3 or more population means. Since we have 3 groups
to test, ANOVA would be the best choice.

\textbf{7. Perform the test. Report the p-value.}

\begin{Shaded}
\begin{Highlighting}[]
\FunctionTok{summary}\NormalTok{(}\FunctionTok{aov}\NormalTok{(lead}\SpecialCharTok{$}\NormalTok{IQ}\SpecialCharTok{\textasciitilde{}}\NormalTok{lead}\SpecialCharTok{$}\NormalTok{GROUP, }\AttributeTok{data=}\NormalTok{lead))}
\end{Highlighting}
\end{Shaded}

\begin{verbatim}
##              Df Sum Sq Mean Sq F value Pr(>F)  
## lead$GROUP    1   1321  1320.6   6.766 0.0102 *
## Residuals   147  28692   195.2                 
## ---
## Signif. codes:  0 '***' 0.001 '**' 0.01 '*' 0.05 '.' 0.1 ' ' 1
\end{verbatim}

\textbf{8. State your conclusion about the evidence for an association
between lead exposure and IQ.}

The p-value is less than 0.05, and so we reject the null hypothesis of
equal means for 3 groups. This means that there is an association
between lead and IQ scores.

\textbf{9. Are there strong reasons to believe that the assumptions of
this test are not met? Briefly justify your answer.}

\begin{Shaded}
\begin{Highlighting}[]
\NormalTok{M }\OtherTok{=} \FunctionTok{with}\NormalTok{(lead, }\FunctionTok{tapply}\NormalTok{(IQ, GROUP, mean))}
\NormalTok{S }\OtherTok{=} \FunctionTok{with}\NormalTok{(lead, }\FunctionTok{tapply}\NormalTok{(IQ, GROUP, sd))}
\NormalTok{N }\OtherTok{=} \FunctionTok{with}\NormalTok{(lead, }\FunctionTok{tapply}\NormalTok{(IQ, GROUP, length))}
\FunctionTok{data.frame}\NormalTok{(M,S,N)}
\end{Highlighting}
\end{Shaded}

\begin{verbatim}
##          M         S  N
## 1 93.72414 15.570313 87
## 2 87.65625  9.502493 32
## 3 86.96667 12.962767 30
\end{verbatim}

From above, we can clearly say that the equal variance assumption is not
met. So, not all the required assumptions for ANOVA are met.

\textbf{10. Conduct all pairwise comparison of group means. Report the
p-values.}

\begin{Shaded}
\begin{Highlighting}[]
\NormalTok{A.vs.B}\OtherTok{=}\FunctionTok{t.test}\NormalTok{(lead}\SpecialCharTok{$}\NormalTok{IQ[lead}\SpecialCharTok{$}\NormalTok{GROUP}\SpecialCharTok{==}\DecValTok{1}\NormalTok{],}
\NormalTok{lead}\SpecialCharTok{$}\NormalTok{IQ[lead}\SpecialCharTok{$}\NormalTok{GROUP}\SpecialCharTok{==}\DecValTok{2}\NormalTok{],}\AttributeTok{var.equal=}\NormalTok{F)}
\NormalTok{A.vs.B}\SpecialCharTok{$}\NormalTok{p.value}
\end{Highlighting}
\end{Shaded}

\begin{verbatim}
## [1] 0.0120506
\end{verbatim}

\begin{Shaded}
\begin{Highlighting}[]
\NormalTok{A.vs.C }\OtherTok{=} \FunctionTok{t.test}\NormalTok{(lead}\SpecialCharTok{$}\NormalTok{IQ[lead}\SpecialCharTok{$}\NormalTok{GROUP}\SpecialCharTok{==}\DecValTok{1}\NormalTok{],}
\NormalTok{lead}\SpecialCharTok{$}\NormalTok{IQ[lead}\SpecialCharTok{$}\NormalTok{GROUP}\SpecialCharTok{==}\DecValTok{3}\NormalTok{],}\AttributeTok{var.equal=}\NormalTok{F)}
\NormalTok{A.vs.C}\SpecialCharTok{$}\NormalTok{p.value}
\end{Highlighting}
\end{Shaded}

\begin{verbatim}
## [1] 0.02300319
\end{verbatim}

\begin{Shaded}
\begin{Highlighting}[]
\NormalTok{B.vs.C }\OtherTok{=} \FunctionTok{t.test}\NormalTok{(lead}\SpecialCharTok{$}\NormalTok{IQ[lead}\SpecialCharTok{$}\NormalTok{GROUP}\SpecialCharTok{==}\DecValTok{2}\NormalTok{],}
\NormalTok{lead}\SpecialCharTok{$}\NormalTok{IQ[lead}\SpecialCharTok{$}\NormalTok{GROUP}\SpecialCharTok{==}\DecValTok{3}\NormalTok{],}\AttributeTok{var.equal=}\NormalTok{F)}
\NormalTok{B.vs.C}\SpecialCharTok{$}\NormalTok{p.value}
\end{Highlighting}
\end{Shaded}

\begin{verbatim}
## [1] 0.8131036
\end{verbatim}

\textbf{11. What conclusion about the association between lead and IQ
would you draw from the pairwise comparisons of group means? Does it
agree with the conclusion in Q8? (Consider the issue of multiple testing
in your answer.)}

Considering the issue of mutltiple testing, we use Bonferroni
correction. And we change the significance level for each individual
test to be 0.05/3 so that the overall significance level is maintained
to 0.05. So, applying that the p-values calculated above are not all
greater than 0.01667, and so we would reject the null hypothesis of
equal means for 3 groups. This means that there is an association
between lead and IQ scores.

\textbf{12. Now we wish to compare the 3 group means for males and
females separately. Display some appropriate descriptive statistics for
this analysis.}

\begin{Shaded}
\begin{Highlighting}[]
\NormalTok{m1 }\OtherTok{=} \FunctionTok{with}\NormalTok{(lead, }\FunctionTok{tapply}\NormalTok{(IQ[lead}\SpecialCharTok{$}\NormalTok{SEX}\SpecialCharTok{==}\DecValTok{1}\NormalTok{], GROUP[lead}\SpecialCharTok{$}\NormalTok{SEX}\SpecialCharTok{==}\DecValTok{1}\NormalTok{], mean))}
\NormalTok{s1 }\OtherTok{=} \FunctionTok{with}\NormalTok{(lead, }\FunctionTok{tapply}\NormalTok{(IQ[lead}\SpecialCharTok{$}\NormalTok{SEX}\SpecialCharTok{==}\DecValTok{1}\NormalTok{], GROUP[lead}\SpecialCharTok{$}\NormalTok{SEX}\SpecialCharTok{==}\DecValTok{1}\NormalTok{], sd))}
\NormalTok{n1 }\OtherTok{=} \FunctionTok{with}\NormalTok{(lead, }\FunctionTok{tapply}\NormalTok{(IQ[lead}\SpecialCharTok{$}\NormalTok{SEX}\SpecialCharTok{==}\DecValTok{1}\NormalTok{], GROUP[lead}\SpecialCharTok{$}\NormalTok{SEX}\SpecialCharTok{==}\DecValTok{1}\NormalTok{], length))}
\NormalTok{m2 }\OtherTok{=} \FunctionTok{with}\NormalTok{(lead, }\FunctionTok{tapply}\NormalTok{(IQ[lead}\SpecialCharTok{$}\NormalTok{SEX}\SpecialCharTok{==}\DecValTok{2}\NormalTok{], GROUP[lead}\SpecialCharTok{$}\NormalTok{SEX}\SpecialCharTok{==}\DecValTok{2}\NormalTok{], mean))}
\NormalTok{s2 }\OtherTok{=} \FunctionTok{with}\NormalTok{(lead, }\FunctionTok{tapply}\NormalTok{(IQ[lead}\SpecialCharTok{$}\NormalTok{SEX}\SpecialCharTok{==}\DecValTok{2}\NormalTok{], GROUP[lead}\SpecialCharTok{$}\NormalTok{SEX}\SpecialCharTok{==}\DecValTok{2}\NormalTok{], sd))}
\NormalTok{n2 }\OtherTok{=} \FunctionTok{with}\NormalTok{(lead, }\FunctionTok{tapply}\NormalTok{(IQ[lead}\SpecialCharTok{$}\NormalTok{SEX}\SpecialCharTok{==}\DecValTok{2}\NormalTok{], GROUP[lead}\SpecialCharTok{$}\NormalTok{SEX}\SpecialCharTok{==}\DecValTok{2}\NormalTok{], length))}

\FunctionTok{data.frame}\NormalTok{(m1,s1,n1,m2,s2,n2)}
\end{Highlighting}
\end{Shaded}

\begin{verbatim}
##         m1       s1 n1       m2        s2 n2
## 1 92.93478 15.42351 46 94.60976 15.877465 41
## 2 90.17647 11.13124 17 84.80000  6.471917 15
## 3 86.61538 17.32791 13 87.23529  8.898942 17
\end{verbatim}

\textbf{13. Perform tests to compare the mean IQ scores in the 3 groups
for males and females separately. Report the p-values from the two
tests.}

\begin{Shaded}
\begin{Highlighting}[]
\FunctionTok{summary}\NormalTok{(}\FunctionTok{aov}\NormalTok{(lead}\SpecialCharTok{$}\NormalTok{IQ[lead}\SpecialCharTok{$}\NormalTok{SEX}\SpecialCharTok{==}\DecValTok{1}\NormalTok{]}\SpecialCharTok{\textasciitilde{}}\NormalTok{lead}\SpecialCharTok{$}\NormalTok{GROUP[lead}\SpecialCharTok{$}\NormalTok{SEX}\SpecialCharTok{==}\DecValTok{1}\NormalTok{], }\AttributeTok{data=}\NormalTok{lead))}
\end{Highlighting}
\end{Shaded}

\begin{verbatim}
##                           Df Sum Sq Mean Sq F value Pr(>F)
## lead$GROUP[lead$SEX == 1]  1    427   427.5   1.942  0.168
## Residuals                 74  16292   220.2
\end{verbatim}

\begin{Shaded}
\begin{Highlighting}[]
\FunctionTok{summary}\NormalTok{(}\FunctionTok{aov}\NormalTok{(lead}\SpecialCharTok{$}\NormalTok{IQ[lead}\SpecialCharTok{$}\NormalTok{SEX}\SpecialCharTok{==}\DecValTok{2}\NormalTok{]}\SpecialCharTok{\textasciitilde{}}\NormalTok{lead}\SpecialCharTok{$}\NormalTok{GROUP[lead}\SpecialCharTok{$}\NormalTok{SEX}\SpecialCharTok{==}\DecValTok{2}\NormalTok{], }\AttributeTok{data=}\NormalTok{lead))}
\end{Highlighting}
\end{Shaded}

\begin{verbatim}
##                           Df Sum Sq Mean Sq F value Pr(>F)  
## lead$GROUP[lead$SEX == 2]  1    922   922.1   5.295 0.0243 *
## Residuals                 71  12366   174.2                 
## ---
## Signif. codes:  0 '***' 0.001 '**' 0.01 '*' 0.05 '.' 0.1 ' ' 1
\end{verbatim}

\textbf{14. What can you conclude about the association between lead and
IQ from these tests? Does it agree with the result in Q8 and Q11?
(Consider multiple testing.)} We see that the null hypothesis of equal
means of three groups for male(1) is not rejected. However, the null
hypothesis of equal means of three groups for female(2) is rejected.
This means that there is an association between lead and IQ for females
and there is no association between lead and IQ for males. When we
performed ANOVA to test the null hypothesis of equal means of three
groups, we reject the null hypothesis Which means that there is an
association between lead and IQ (Q8) When we performed pairwise test to
test the null hypothesis of equla means of three groups, we reject the
null hypothesis which means that there is an association between lead
and IQ (Q11)

\textbf{15. Now perform all 3 pairwise comparisons of groups for males
and females separately. Report the p-values from these tests?}

\begin{Shaded}
\begin{Highlighting}[]
\NormalTok{Am.vs.Bm }\OtherTok{=} \FunctionTok{t.test}\NormalTok{(lead}\SpecialCharTok{$}\NormalTok{IQ[lead}\SpecialCharTok{$}\NormalTok{GROUP}\SpecialCharTok{==}\DecValTok{1} \SpecialCharTok{\&}\NormalTok{ lead}\SpecialCharTok{$}\NormalTok{SEX}\SpecialCharTok{==}\DecValTok{1}\NormalTok{],}
\NormalTok{lead}\SpecialCharTok{$}\NormalTok{IQ[lead}\SpecialCharTok{$}\NormalTok{GROUP}\SpecialCharTok{==}\DecValTok{2} \SpecialCharTok{\&}\NormalTok{ lead}\SpecialCharTok{$}\NormalTok{SEX}\SpecialCharTok{==}\DecValTok{1}\NormalTok{],}\AttributeTok{var.equal=}\NormalTok{F)}
\NormalTok{Am.vs.Bm}\SpecialCharTok{$}\NormalTok{p.value}
\end{Highlighting}
\end{Shaded}

\begin{verbatim}
## [1] 0.4391938
\end{verbatim}

\begin{Shaded}
\begin{Highlighting}[]
\NormalTok{Am.vs.Cm }\OtherTok{=} \FunctionTok{t.test}\NormalTok{(lead}\SpecialCharTok{$}\NormalTok{IQ[lead}\SpecialCharTok{$}\NormalTok{GROUP}\SpecialCharTok{==}\DecValTok{1} \SpecialCharTok{\&}\NormalTok{ lead}\SpecialCharTok{$}\NormalTok{SEX}\SpecialCharTok{==}\DecValTok{1}\NormalTok{],}
\NormalTok{lead}\SpecialCharTok{$}\NormalTok{IQ[lead}\SpecialCharTok{$}\NormalTok{GROUP}\SpecialCharTok{==}\DecValTok{3} \SpecialCharTok{\&}\NormalTok{ lead}\SpecialCharTok{$}\NormalTok{SEX}\SpecialCharTok{==}\DecValTok{1}\NormalTok{],}\AttributeTok{var.equal=}\NormalTok{F)}
\NormalTok{Am.vs.Cm}\SpecialCharTok{$}\NormalTok{p.value}
\end{Highlighting}
\end{Shaded}

\begin{verbatim}
## [1] 0.2502756
\end{verbatim}

\begin{Shaded}
\begin{Highlighting}[]
\NormalTok{Bm.vs.Cm }\OtherTok{=} \FunctionTok{t.test}\NormalTok{(lead}\SpecialCharTok{$}\NormalTok{IQ[lead}\SpecialCharTok{$}\NormalTok{GROUP}\SpecialCharTok{==}\DecValTok{2} \SpecialCharTok{\&}\NormalTok{ lead}\SpecialCharTok{$}\NormalTok{SEX}\SpecialCharTok{==}\DecValTok{1}\NormalTok{],}
\NormalTok{lead}\SpecialCharTok{$}\NormalTok{IQ[lead}\SpecialCharTok{$}\NormalTok{GROUP}\SpecialCharTok{==}\DecValTok{3} \SpecialCharTok{\&}\NormalTok{ lead}\SpecialCharTok{$}\NormalTok{SEX}\SpecialCharTok{==}\DecValTok{1}\NormalTok{],}\AttributeTok{var.equal=}\NormalTok{F)}
\NormalTok{Bm.vs.Cm}\SpecialCharTok{$}\NormalTok{p.value}
\end{Highlighting}
\end{Shaded}

\begin{verbatim}
## [1] 0.5258587
\end{verbatim}

\begin{Shaded}
\begin{Highlighting}[]
\NormalTok{Af.vs.Bf }\OtherTok{=} \FunctionTok{t.test}\NormalTok{(lead}\SpecialCharTok{$}\NormalTok{IQ[lead}\SpecialCharTok{$}\NormalTok{GROUP}\SpecialCharTok{==}\DecValTok{1} \SpecialCharTok{\&}\NormalTok{ lead}\SpecialCharTok{$}\NormalTok{SEX}\SpecialCharTok{==}\DecValTok{2}\NormalTok{],}
\NormalTok{lead}\SpecialCharTok{$}\NormalTok{IQ[lead}\SpecialCharTok{$}\NormalTok{GROUP}\SpecialCharTok{==}\DecValTok{2} \SpecialCharTok{\&}\NormalTok{ lead}\SpecialCharTok{$}\NormalTok{SEX}\SpecialCharTok{==}\DecValTok{2}\NormalTok{],}\AttributeTok{var.equal=}\NormalTok{F)}
\NormalTok{Af.vs.Bf}\SpecialCharTok{$}\NormalTok{p.value}
\end{Highlighting}
\end{Shaded}

\begin{verbatim}
## [1] 0.001830775
\end{verbatim}

\begin{Shaded}
\begin{Highlighting}[]
\NormalTok{Af.vs.Cf }\OtherTok{=} \FunctionTok{t.test}\NormalTok{(lead}\SpecialCharTok{$}\NormalTok{IQ[lead}\SpecialCharTok{$}\NormalTok{GROUP}\SpecialCharTok{==}\DecValTok{1} \SpecialCharTok{\&}\NormalTok{ lead}\SpecialCharTok{$}\NormalTok{SEX}\SpecialCharTok{==}\DecValTok{2}\NormalTok{],}
\NormalTok{lead}\SpecialCharTok{$}\NormalTok{IQ[lead}\SpecialCharTok{$}\NormalTok{GROUP}\SpecialCharTok{==}\DecValTok{3} \SpecialCharTok{\&}\NormalTok{ lead}\SpecialCharTok{$}\NormalTok{SEX}\SpecialCharTok{==}\DecValTok{2}\NormalTok{],}\AttributeTok{var.equal=}\NormalTok{F)}
\NormalTok{Af.vs.Cf}\SpecialCharTok{$}\NormalTok{p.value}
\end{Highlighting}
\end{Shaded}

\begin{verbatim}
## [1] 0.02927457
\end{verbatim}

\begin{Shaded}
\begin{Highlighting}[]
\NormalTok{Bf.vs.Cf }\OtherTok{=} \FunctionTok{t.test}\NormalTok{(lead}\SpecialCharTok{$}\NormalTok{IQ[lead}\SpecialCharTok{$}\NormalTok{GROUP}\SpecialCharTok{==}\DecValTok{2} \SpecialCharTok{\&}\NormalTok{ lead}\SpecialCharTok{$}\NormalTok{SEX}\SpecialCharTok{==}\DecValTok{2}\NormalTok{],}
\NormalTok{lead}\SpecialCharTok{$}\NormalTok{IQ[lead}\SpecialCharTok{$}\NormalTok{GROUP}\SpecialCharTok{==}\DecValTok{3} \SpecialCharTok{\&}\NormalTok{ lead}\SpecialCharTok{$}\NormalTok{SEX}\SpecialCharTok{==}\DecValTok{2}\NormalTok{],}\AttributeTok{var.equal=}\NormalTok{F)}
\NormalTok{Bf.vs.Cf}\SpecialCharTok{$}\NormalTok{p.value}
\end{Highlighting}
\end{Shaded}

\begin{verbatim}
## [1] 0.3796352
\end{verbatim}

\textbf{16. What do you conclude about the association between lead and
IQ from the results in Q13? Does your conclusion change from previous
conclusions made in Q8, Q11 and Q14?}

From Q15, the p-values for pairwise test for three groups for males are
greater than 0.05/3, so we do not have enough evidence to reject the
null hypothesis of equal means for three groups for males at 0.05
significance level. This means that there is no association between lead
and IQ for males. However, the p-values for pairwise test for three
groups for females are not all greater than 0.05/3, so we reject the
null hypothesis of equal means for three groups for females at 0.05
significance level. This means that there is an association between lead
and IQ for females. From Q14, we arrived at a similar conclusion where,
there is an association between lead and IQ for females and there is no
association between lead and IQ for males. From Q11, we arrived at a
result that there is an association between lead and IQ. From Q8, we
arrived at a result that there is an association between lead and IQ.
So, our conclsuion changes from previous questions for males, however it
doesn't change for females.

\end{document}
